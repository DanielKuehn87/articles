


\documentclass[10pt]{article}

\frenchspacing

\usepackage[english,dutch]{babel}

\usepackage{graphicx}
\usepackage{hyperref}

\usepackage{listings}
\lstset{language=C++, showstringspaces=false, basicstyle=\small,
  numbers=left, numberstyle=\tiny, numberfirstline=false,
  stepnumber=1, tabsize=8, 
  commentstyle=\ttfamily, identifierstyle=\ttfamily,
  stringstyle=\itshape}

\author{The OpenML Team}
\title{OpenML --- Technical details}

\begin{document}

\selectlanguage{english}

\maketitle
\section{Introduction} 

The OpenML webplatform allows researchers to share, organize and highlight the results of their experiments. OpenML is written in PHP and build on top of CodeIgniter, an open source webdevelopment framework. In this document we will explain how to setup a local version on the system, how it works, and how to extend it. 

\subsection{Setup}

*This will follow, no priority.*

\section{Important files and folders}
In this section we go through all important files and folder of the system.

\subsection{Root directory}
The root directory of OpenML contains the following files and folders. 
\begin{itemize}
\item \textbf{system}: This folder contains all files provided by CodeIgniter~2.1.3~\cite{bib:codeigniter}. The contents of this folder is beyond the scope of this document, and not relevant for extending OpenML. All the files in this folder are in the same state as they were provided by Ellislabs, and none of these files should ever be changed. 
\item \textbf{sparks}: Sparks~\cite{bib:sparks} is a package management system for Codeigniter that allows for instant installation of libraries into the application. This folder contains two libraries provided by third party software developers, oauth1 (based on version 1 the oauth protocol) and oauth2 (similarly, based on version 2 of the oauth protocol). The exact contents of this folder is beyond the scope of this document and not relevant for extending OpenML.
\item \textbf{openml\_OS}: All files in this folder are written specifically for OpenML. When extending the functionality OpenML, usually one of the files in this folder needs to be adjusted. As a thorough understanding of the contents of this folder is vital for extending OpenML, we will discuss the contents of this folder in Section~\ref{sec:openmlos} in more detail.
\item \textbf{index.php}: This is the ``bootstrap'' file of the system. Basically, every page request on OpenML goes through this file (with the css, images and javascript files as only exception). It then determines which CodeIgniter and OpenML files need to be included. This file should not be edited. 
\item \textbf{.htaccess}: This file (which configures the Apache Rewrite Engine) makes sure that all URL requests will be directed to \texttt{index.php}. Without this file, we would need to include \texttt{index.php} explicitly in every URL request. This file makes sure that all other URL requests without \texttt{index.php} embedded in it automatically will be transformed to \texttt{index.php}. Eg., \url{http://www.openml.org/frontend/page/home} will be rewritten to \url{http://www.openml.org/index.php/frontend/page/home}. This will be explained in detail in Section~\ref{sec:urlshortening}.
\item \textbf{css}: A folder containing all stylesheets. These are important for the layout of OpenML.
\item \textbf{data}: A folder containing data files, e.g., datasets, implementation files, uploaded content. Please note that this folder does not necessarily needs to be present in the root directory. The OpenML Base Config file (Section~\ref{sec:openmlos}) determines the exact location of this folder. 
\item \textbf{downloads}: Another data folder, containing files like the most recent database snapshot. 
\item \textbf{img}: A folder containing all static images shown on the webpage.
\item \textbf{js}: A folder containing all used Javascript files and libraries, including third party libraries like jQuery and datatables. 
\item Various other files, like .gitignore, favicon.ico, etc. 
\end{itemize}

\subsection{openml\_OS}
\label{sec:openmlos}
This folder is (in CodeIgniter jargon) the ``Application folder'', and contains all files relevant to OpenML. Within this folder, the following folders should be present: (And also some other folders, but these are not used by OpenML)
\begin{itemize}
\item \textbf{config}: A folder containing all config files. Most notably, it contains the file {\tt BASE\_CONFIG.php}, in which all system specific variables are set; the config items within this file differs over various installations (e.g., on localhost, \texttt{openml.org}). Most other config files, like {\tt database.php}, will receive their values from {\tt BASE\_CONFIG.php}. Other important config files are {\tt autoload.php}, determining which CodeIgniter / OpenML files will be loaded on any request, {\tt openML.php}, containing config items specific to OpenML, and {\tt routes.php}, which will be explained in Section~\ref{sec:urlshortening}.
\item \textbf{controllers}: In the Model/View/Controller design pattern, all user interaction goes through controllers. In a webapplication setting this means that every time a URL gets requested, exactly one controller gets invoked. The exact dynamics of this will be explained in Section~\ref{sec:routes}. 
\item \textbf{core}: A folder that contains CodeIgniter specific files. These are not relevant for the understanding of OpenML. 
\item \textbf{helpers}: This folder contains many convenience functions. Wikipedia states: ``A convenience function is a non-essential subroutine in a programming library or framework which is intended to ease commonly performed tasks''. For example the {\tt file\_upload\_helper.php} contains many functions that assist with uploading of files. Please note that a helper function must be explicitly loaded in either the autoload config or the files that uses its functions. 
\item \textbf{libraries}: Similar to sparks, this folder contains libraries specifically written for CodeIgniter. For example, the library used for all user management routines is in this folder.
\item \textbf{models}: In the Model/View/Controller design pattern, models represent the state of the system. In a webapplication setting, you could say that a model is the link to the database. In OpenML, almost all tables of the database are represented by a model. Each model has general functionality applicable to all models (e.g., retrieve all records, retrieve record with constraints, insert record) and functionality specific to that model (e.g., retrieve a dataset that has certain data properties). Most models extend an (abstract) base class, located in the {\tt abstract} folder. This way, all general functionality is programmed and maintained in one place. 
\item \textbf{third\_party}: Although the name might suggests differently, this folder contains all OpenML Java libraries. 
\item \textbf{views}: In the Model/View/Controller design pattern, the views are the way information is presented on the screen. In a webapplication setting, a view usually is a block of (PHP generated) HTML code. The most notable view is {\tt frontend\_main.php}, which is the template file determining the main look and feel of OpenML. Every single page also has its own specific view (which is parsed within {\tt frontend\_main.php}). These pages can be found (categorized by controller and name) in the {\tt pages} folder. More about this structure is explained in Section~\ref{sec:routes}.
\end{itemize}

\section{URL to Page Mapping}
\label{sec:routes}
Most pages in OpenML are represented by a folder in 
$$\texttt{\textless root\_directory\textgreater /openml\_OS/views/pages/frontend}$$
The contents of this folder will be parsed in the template  \texttt{frontend\_main.php} template, as described in Section~\ref{sec:openmlos}. In this section we explain the way an URL is mapped to a certain OpenML page. 

\subsection{URL Anatomy}
By default, CodeIgniter (and OpenML) accepts a URL in the following form: 
$$\texttt{http://www.openml.org/index.php/\textless controller\textgreater/\textless function\textgreater/\textless p1\textgreater/\textless pN\textgreater/\textless free\textgreater}$$ 
The various parts in the URL are divided by slashes. Every URL starts with the protocol and server name (in the case of OpenML this is \texttt{http://www.openml.org/}). This is followed by the bootstrap file, which is always the same, i.e., \texttt{index.php}. The next part indicates the controller that needs to be invoked; typically this is \texttt{frontend}, \texttt{rest\_api} or \texttt{data}, but it can be any file from the \texttt{openml\_OS} folder \texttt{controllers}. Note that the suffix \texttt{.php} should not be included in the URL. 

The next part indicates which function of the controller should be invoked. This should be a existing, public function from the controller that is indicated in the controller part. These functions might have one or more parameters that need to be set. This is the following part of the URL (indicated by \texttt{p1} and \texttt{pN}). The parameters can be followed by anything in free format. Typically, this free format is used to pass on additional parameters in \texttt{name} - \texttt{value} format, or just a way of adding a human readable string to the URL for SEO purposes. 

For example, the following URL
$$\texttt{http://www.openml.org/index.php/frontend/page/home}$$
invokes the function \texttt{page} from the \texttt{frontend} controller and sets the only parameter of this function, \texttt{\$indicator}, to value \texttt{home}. The function \texttt{page} loads the content of the specified folder (\texttt{\$indicator}) into the main template. In this sense, the function \texttt{page} can be seen as some sort of specialized page loader. 

\subsection{URL Shortening}
\label{sec:urlshortening}
Since it is good practice to have URL's as short as possible, we have introduced some logic that shortens the URL's. Most importantly, the URL part that invokes \texttt{index.php} can be removed at no cost, since this file is \textbf{always} invoked. For this, we use Apache's rewrite engine. Rules for rewriting URL's can be found in the \texttt{.htaccess} file, but is suffices to say that any URL in the following format
$$\texttt{http://www.openml.org/index.php/\textless controller\textgreater /\textless function\textgreater /\textless params\textgreater}$$
can due to the rewrite engine also be requested with
$$\texttt{http://www.openml.org/\textless controller\textgreater /\textless function\textgreater /\textless params\textgreater}$$

Furthermore, since most of the pages are invoked by the function \texttt{page} of the \texttt{frontend} controller (hence, they come with the suffix \texttt{frontend/page/page\_name}) we also created a mapping that maps URL's in the following form
$$\texttt{http://www.openml.org/\textless page\_name\textgreater}$$
to
$$\texttt{http://www.openml.org/frontend/page/\textless page\_name\textgreater}$$
Note that Apache's rewrite engine will also add \texttt{index.php} to this. The exact mapping can be found in \texttt{routes.php} config file.

\subsection{Additional Mappings}
Additionally, a mapping is created from the following type of URL:
$$\texttt{http://www.openml.org/api/\textless any\_query\_string\textgreater}$$
to 
$$\texttt{http://www.openml.org/rest\_api/\textless any\_query\_string\textgreater}$$
This was done for backwards compatibility. Many plugins make calls to the not-existing \texttt{api} controller, which are automatically redirected to the \texttt{rest\_api} controller. 

\subsection{Exceptions}
It is important to note that not all pages do have a specific page folder. The page folders are a good way of structuring complex GUI's that need to be presented to the user, but in cases where the internal state changes are more important than the GUI's, it might be preferable to make the controller function print the output directly. This happens for example in the functions of \texttt{rest\_api.php} and \texttt{free\_query.php} (although the former still has some files in the views folder that it refers to). 

\section{Create new page}
Based on Section~\ref{sec:routes}, it should be clear how a new page can be added to OpenML. One method of doing this is adding an additional function to one of the controllers; this page can later be requested by $$\texttt{http://www.openml.org/\textless controller\textgreater /\textless function\textgreater /\textless parameters\textgreater}$$
This method is preferable for API functions, where the internal actions are complex, but the output is simple. 

The other method is creating a new folder into the folder
$$\texttt{\textless root\_directory\textgreater /openml\_OS/views/pages/frontend}$$
This page can be requested by
$$\texttt{http://www.openml.org/frontend/page/\textless folder\_name\textgreater}$$
or just
$$\texttt{http://www.openml.org/\textless folder\_name\textgreater}$$
This method is preferred for human readable webpages, where the internal actions are simple, and the output is complex. We will describe the files that can be in this folder.
\begin{itemize}
  \item \textbf{pre.php}: Mandatory file. Will be executed first. Do not make this file produce any output! Can be used to pre-render data, or set some variables that are used in other files. 
  \item \textbf{body.php}: Highly recommended file. Intended for displaying the main content of this file. Will be rendered at the right location within the template file (\texttt{frontend\_main.php}).
  \item \textbf{javascript.php}: Non-mandatory file. Intended for javascript function on which \texttt{body.php} relies. Will be rendered within a javascript block in the header of the page.
  \item \textbf{post.php}: Non mandatory file. Will only be executed when a POST request is done (e.g., when a HTML form was send using the POST protocol).  Will be executed after \texttt{pre.php}, but before the rendering process (and thus, before \texttt{body.php} and \texttt{javascript.php}). Should handle the posted input, e.g., file uploads. 
\end{itemize}
It is also recommended to add the newly created folder to the mapping in the \texttt{routes.php} config file. This way it can also be requested by the shortened version of the URL. (Note that we deliberately avoided to auto-load all pages into this file using a directory scan, as this makes the webplatform slow. )


\begin{thebibliography}{9}

\bibitem{bib:codeigniter}
	\url{http://ellislab.com/codeigniter/user-guide/} [retrieved: March 11, 2014]

\bibitem{bib:sparks}
	\url{http://getsparks.org/} [retrieved: March 11, 2014]

\end{thebibliography}

\end{document}